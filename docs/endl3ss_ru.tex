\documentclass[a4paper,12pt,fleqn]{article}
\usepackage{polyglossia} 
\setdefaultlanguage[babelshorthands=true]{russian}
\setotherlanguage{english}
\defaultfontfeatures{Ligatures=TeX,Mapping=tex-text}

\setmainfont{Linux Libertine}

\usepackage{xltxtra}
\usepackage{microtype}

\usepackage{xcolor}
\definecolor{darkblue}{HTML}{003153}
\usepackage{graphicx}
\usepackage {textcomp}
\usepackage[twoside,left=2.5cm,right=3cm,top=3cm,bottom=4cm,bindingoffset=0cm]{geometry}

\newcommand{\spacing}{\textcolor{red}{\textbf{(разрыв)}}}

\usepackage{amsmath}
\usepackage{amssymb}
\usepackage{amsfonts}

\usepackage{hyperref}
\hypersetup{colorlinks=true, linkcolor=darkblue, citecolor=darkblue, filecolor=darkblue, urlcolor=darkblue}

\usepackage{fancyhdr}
\pagestyle{fancy}
\fancyhead[LE,RO]{\thepage}
\fancyhead[LO]{{\small ENDL3SS}}
\fancyhead[RE]{{\small ДОКУМЕНТАЦИЯ}} 
\fancyfoot{}
\fancypagestyle{plain}
{
\fancyhead{}
\renewcommand{\headrulewidth}{0mm}
\fancyfoot{}
}

\begin{document}

\begin{titlepage}
{\centering
{~\par}
\vspace{0.25\textheight}
{\Huge 
\begin{minipage}{0.4\textwidth}
\[\notin\mathbb{N},\frac{\partial l}{\partial x}\sum\iint\]
\end{minipage}
\par}
\vspace{1.0cm}
{\Large Документация \par}
\vfill
{\Large Создано в \XeLaTeX }\par}
\end{titlepage}

\tableofcontents

\section{Sky}

\subsection{SphericRandom}

В ходе разработки модуля \verb|Sky| пришлось решать задачу равномерного распределения звёзд по небу.
В высшей математике я не силён, поэтому сразу стал искать решение в интернете.
Наткнулся на \href{https://www.fundamental-research.ru/ru/article/view?id=31243}{статью} под авторством Копытова и Митюшова.

Польза этой статьи в том, что там рассматривалось общее решение равномерного распределения точек по поверхностям.
В частности, среди примеров были тор и бутылка Клейна.

Итак, для поверхности, заданной параметрическими уравнениями \[x = x(u, v), y = y(u, v), z = z(u, v)\] на области \[D = \left \{ u_1 \leqslant u \leqslant u_2; v_1 \leqslant v \leqslant v_2 \right \}\] функция равномерного распределения выглядит так:

\[f(u, v) = \frac{\sqrt{EG - F^2}}{\iint\limits_D \sqrt{EG - F^2}\,du\,dv}\]

$E$, $G$ и $F$ "--- коэффициенты первой квадратичной формы поверхности.

Для сферы, задаваемой параметрами

\[r'_u(- a \sin u \cos v; - a \sin u \sin v; - a \cos u),\]
\[r'_v(-a \cos u \sin v; a \cos u \cos v; 0)\]

на области

\[D = \left \{ 0 \leqslant u \leqslant \pi; 0 \leqslant v \leqslant 2\pi \right \}\]

коэффициенты п.к.ф. равны

\[E = a^2, F = 0, G = a^2 \cos^2 u\]

Для генерации случайных чисел на основе известной функции распределения использовался метод Неймана:

\begin{enumerate}
\item Функция плотности распределения вписывается в прямоугольник;
\item Генерируются два независимых числа эталонным генератором случайной величины с равномерным распределением на интервале $(0, 1)$ и масштабируются по сторонам прямоугольника;
\item Если полученная точка попадает в область под графиком плотности распределения, то точка принимается, иначе отбрасывается;
\item Повторяются действия 2--3 до тех пор, пока не будет сгенерировано необходимое число точек.                                                                                            \end{enumerate}

После ряда упрощений я пришёл к алгоритму, описанному в функции \verb|SphericRandom|.
Он возвращает единичный вектор на основе генератора случайных чисел.
Эмпирически алгоритм соответствует поставленной задаче, тестировался при $N = 10000$.

\section{Weather}

Эта система создаёт изменения погоды, напоминающие реальные.

В качестве входных данных используются следующие:

\begin{itemize}
\item Широта, на которой находится наблюдатель ($\theta$);
\item Время года, сиречь отношение наклона оси к месту планеты на орбите ($\alpha$).                                                    \end{itemize}

Воздушные потоки определяются циклонами и антициклонами.
Эти два типа объектов генерируются независимо друг от друга.

Радиус объекта, он же "--- время жизни.
Чем дальше объект от экватора, тем больше его радиус:

\[r = | \sin \theta |\]

Функции вероятности.
Для циклона и антициклона взяты следующие функции вероятности:

\[f_c = \cos \theta \left( 1 - \frac{1}{4} \sum r_i^2 \right ) | \sin \alpha | \cdot | \sin \theta |\]

\[f_a = \cos \theta \left( 1 - \frac{1}{4} \sum r_i^2 \right ) | \cos \alpha | \cdot | \cos 2 \theta |\]

Компоненты функций:

\begin{enumerate}
\item Компонент равномерного распределения на сфере;
\item Компонент, уменьшающий вероятность генерации в зависимости от заполненности поверхности сферы;
\item Компонент, увеличивающий вероятность генерации циклона в межсезонье, а антициклона "--- зимой или летом;
\item Компонент, увеличивающий вероятность генерации циклона ближе к полюсу, а антициклона "--- на полюсах и экваторе.                                                                                                                      \end{enumerate}

Далее генерация объекта происходит по уже ипользовавшемуся методу Неймана.

\section{Console}

\end{document}
 
